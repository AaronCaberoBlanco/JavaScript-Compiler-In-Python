 
\documentclass[11pt, , a4paper, titlepage]{article}
\setlength{\parindent}{0pt} % quitamos indentacion automática en parrafos

% \addtolength{\textwidth}{+100cm}

\usepackage[utf8]{inputenc}
\usepackage[T1]{fontenc}
\usepackage[spanish]{babel}
\usepackage{float}
% \usepackage{lmodern}
\usepackage{blindtext}
\usepackage{vmargin}
\usepackage[hidelinks]{hyperref}
\usepackage{url}
\usepackage{lipsum}
\usepackage{adforn} % símbolo lista
\usepackage{graphicx}

% PA LA PROXIMA, HAY QUE TENERLO EN CUENTA DESDE EL PRINCIPIO 
% \setmargins{3cm}       % margen izquierdo
% {1cm}                        % margen superior
% {16.5cm}                      % anchura del texto
% {23.42cm}                    % altura del texto
% {10pt}                           % altura de los encabezados
% {2cm}                           % espacio entre el texto y los encabezados
% {0pt}                             % altura del pie de página
% {1.5cm}                           % espacio entre el texto y el pie de página


\usepackage[dvipsnames]{xcolor}
\usepackage{csquotes}   
\usepackage[style=numeric-comp, sorting=none, block=par]{biblatex} 
\addbibresource{references.bib}
\DeclareFieldFormat{title}{\bfseries\emph{#1}}
\definecolor{softblack}{RGB}{74, 71, 71} 
\DeclareFieldFormat{howpublished}{\textcolor{softblack}{\mdseries{#1}}}
\DeclareFieldFormat{labelnumberwidth}{\mkbibbold{#1\adddot}}
\setlength\bibitemsep{2.5\itemsep}
\renewcommand*{\newunitpunct}{\addspace}
\renewcommand*{\finentrypunct}{\addspace}
% \renewcommand\mkbibnamefamily[1]{\textbf{#1}}

% \usepackage{xurl} % tiene que estar despues de biblatex

\usepackage{tocloft}
% \setlength{\cftbeforesecskip}{6pt}
\setlength{\cftbeforesubsecskip}{6pt}
\setlength{\cftbeforesubsubsecskip}{3pt}


\usepackage{titlesec}
\definecolor{gray75}{gray}{0.75}
\newcommand{\hsp}{\hspace{20pt}}
\titleformat{\section}[hang]{\LARGE\bfseries}{\thesection\hsp\textcolor{gray75}{|}\hsp}{0pt}{\LARGE\bfseries}
\titlespacing{\section}{0pt}{0pt}{15pt}
\titleformat{\subsection}[hang]{\Large\bfseries}{\thesubsection\hsp}{0pt}{\Large\bfseries}
\titlespacing{\subsection}{0pt}{35pt}{15pt}
\titleformat{\subsubsection}[hang]{\large\bfseries}{\thesubsubsection\hspace{10pt}}{0pt}{\large\bfseries} 
\titlespacing{\subsubsection}{0pt}{20pt}{0pt}


\newenvironment{myitemize}
{ \begin{itemize}
    \setlength{\itemsep}{0pt}
    \setlength{\parskip}{2pt}    }
{ \end{itemize}                  } 
    
\newcommand{\minus}{\scalebox{0.75}[1.0]{$-$}}
    
\newenvironment{changemargin}[2]{%
\begin{list}{}{%
\setlength{\topsep}{0pt}%
\setlength{\leftmargin}{#1}%
\setlength{\rightmargin}{#2}%
\setlength{\listparindent}{\parindent}%
\setlength{\itemindent}{\parindent}%
\setlength{\parsep}{\parskip}%
}%
\item[]}{\end{list}}



\renewcommand*{\ttdefault}{\familydefault}

\usepackage{fancyvrb}

% redefine \VerbatimInput
\RecustomVerbatimCommand{\VerbatimInput}{VerbatimInput}%
{fontfamily=cmr, % no hace falta si se cambia ttdefault
 codes={\catcode`$=3},
 commandchars=\#\[\] % escape character and argument delimiters for
                       % commands within the verbatim
}
  
\usepackage{booktabs}
\usepackage{listings}
  
\lstdefinelanguage{JavaScript}{
  morekeywords=[1]{break, continue, delete, else, for, function, if, in,
    new, return, this, typeof, var, void, while, with, let},
  % Literals, primitive types, and reference types.
  morekeywords=[2]{false, null, true, boolean, number, string, undefined,
    Array, Boolean, Date, Math, Number, String, Object},
  % Built-ins. 
  morekeywords=[3]{eval, parseInt, parseFloat, escape, unescape},
  sensitive,
  morecomment=[s]{/*}{*/},
  morecomment=[l]//,
  morecomment=[s]{/**}{*/}, % JavaDoc style comments
  morestring=[b]',
  morestring=[b]"
}[keywords, comments, strings]

\definecolor{mediumgray}{rgb}{0.3, 0.4, 0.4}
\definecolor{mediumblue}{rgb}{0.0, 0.0, 0.8}
\definecolor{forestgreen}{rgb}{0.13, 0.55, 0.13}
\definecolor{darkviolet}{rgb}{0.58, 0.0, 0.83}
\definecolor{royalblue}{rgb}{0.25, 0.41, 0.88}
\definecolor{crimson}{rgb}{0.86, 0.8, 0.24}

\lstdefinestyle{JSES6Base}{
    backgroundcolor=\color{white},
    basicstyle=\ttfamily,
    breakatwhitespace=false,
    breaklines=false,
    captionpos=b,
    columns=fullflexible,
    commentstyle=\color{mediumgray}\upshape,
    emph={},
    emphstyle=\color{crimson},
    extendedchars=true,  % requires inputenc
    literate={á}{{\'a}}1 {é}{{\'e}}1 {í}{{\'i}}1 {ó}{{\'o}}1 {ú}{{\'u}}1,
    fontadjust=true,
    frame=single,
    identifierstyle=\color{black},
    keepspaces=true,
    keywordstyle=\color{mediumblue},
    keywordstyle={[2]\color{darkviolet}},
    keywordstyle={[3]\color{royalblue}},
    numbers=none,
    rulecolor=\color{black},
    showlines=false,
    showspaces=false,
    showstringspaces=false,
    showtabs=false,
    stringstyle=\color{forestgreen},
    tabsize=2,
    upquote=true,  % requires textcomp
    numbers=left, numberstyle=\scriptsize
}

\lstdefinestyle{JavaScript}{
  language=JavaScript,
  style=JSES6Base
}


% \usepackage{geometry}
%  \geometry{
%  a4paper,
%  total={210mm,297mm},
%  left=60mm,
%  right=20mm,
%  top=40mm,
%  bottom=20mm,
%  }
 
% \usepackage{fancyhdr}

% \pagestyle{fancy}
% \fancyhead[LO,LE]{Curso 2020/2021}
% \fancyhead[CO,CE]{Grado en Ingeniería Informática}
% \fancyfoot[CO,CE]{\thepage}

% \renewcommand{\headrulewidth}{0.4pt} % grosor de la línea de la cabecera
% \renewcommand{\footrulewidth}{0.4pt} % grosor de la línea del pie
% \usepackage{pdfpages} % to include external pdf's

% \setpapersize{A4}
% \setmargins{2.5cm}       % margen izquierdo
% {1cm}                        % margen superior
% {16.5cm}                      % anchura del texto
% {23.42cm}                    % altura del texto
% {10pt}                           % altura de los encabezados
% {2cm}                           % espacio entre el texto y los encabezados
% {0pt}                             % altura del pie de página
% {2cm}                           % espacio entre el texto y el pie de página

% \usepackage{geometry}
%  \geometry{
%  a4paper,
%  total={170mm,257mm},
%  left=20mm,
%  top=20mm,
%  }

% \title{Ejercicio Individual II}
% \author{Daniel Tomás Sánchez\\ asdas\\ dfsf}
% \date{Diciembre 2020}

\begin{document}
% Cuerpo del documento
\begin{titlepage}
    \begin{center}
        \hrulefill

        \vspace{0.5cm}
        {\bfseries\Huge Traductores de Lenguajes \par}
        \vspace{3cm}

        {\scshape \LARGE \textbf{Memoria Final}}

        \hrulefill

        \vspace{2.0cm}
    \end{center}

    \centering

    % {\scshape\Huge\textbf{}\par}
    % \vspace{1cm}
    % {\scshape\Huge\textbf{Ejercicio Individual II}\par}
    % \vspace{3cm}

    \Large{Grupo 55}\\
    \vspace{0.3cm}

    {\large Daniel Tomás Sánchez\\ Aarón Cabero Blanco\\ Alejandro Cuadrón Lafuente\par}

    \vspace{2cm}
    {\Large Curso 2020/2021 \par}
\end{titlepage}

%\maketitle

% \clearpage
\tableofcontents
\clearpage


\section{Introducción}
Al ser esta práctica continuación directa de la creada en PDL, nos mantuvimos usando Python y la herramienta externa "SLY" \cite{SLY}. \\
Sin embargo, la herramienta la mantuvimos por compatibilidad con la parte anterior. Para la realización de esta parte de la práctica no empleamos ninguna herramienta o libería adicional.\\

Opciones de grupo:
\begin{myitemize}
    \renewcommand{\labelitemi}{$\circ$}
    \item Sentencias: Sentencia repetitiva (\textbf{for})
    \item   Operadores especiales: Post-auto-decremento (\textbf{\minus \minus \hspace{0.1cm} como sufijo})
    \item     Técnicas de Análisis Sintáctico: \textbf{Ascendente}
    \item     Comentarios: Comentario de bloque (\textbf{/* */})
    \item Cadenas: Con comillas dobles (\textbf{" \phantom{} "})
\end{myitemize}
\clearpage

\section{Diseño del generador de código intermedio}

% % LO SUYO hubiese sido hacerlo asi
% \newcommand{\fila}[2]{$\bullet$ & #1 & <#2>\\}
% \begin{tabular}{cll}
% \fila{Identificador}{ID, punteroTS}
% \fila{Palabra reservada Number}{<NUMBER, ->}
% \end{tabular}

Para diseñar el generador de código intermedio nos basamos en la gramática empleada en la práctica de PDL.
El diseño resultante es el siguiente: \\ \\

\VerbatimInput{GCI2.txt}

\section{Diseño del registro de activación}
El registro de activación considerado para la práctica es el siguiente: \\
\begin{table}[H]
    \centering
    \begin{tabular}{ l c }
    Estado de la máquina & EM \\
    Parámetros actuales  & P  \\
    Variables locales    & VL \\
    Datos temoporales    & DT \\
    Valor devuelto       & VD
    \end{tabular}
\end{table}

Se han suprimido tanto el PC como el PA. El primero se debe a que los diferentes RAs se almacenan
en pila uno encima del anterior, por lo tanto no es necesario almacenar la dirección del antiguo RA que llamó a este.
El segundo se ha suprimido porque al no haber funciones anidadas en nuestro lenguaje no hay necesidad de almacenar un PA. \\ \\
El campo valor devuelto (VD), se ha implementado en la práctica usando un registro. Es decir, en lugar de extraer el resultado de la función
del campo VD, se almacena en un registro y de este registro se extrae dicho valor. Más adelante mostrará en CO su uso.

\clearpage

\section{Diseño del código objeto}

\subsection{Métodos auxiliares}
\subsubsection{Store\_in\_reg}
\subsubsection{Copy\_str}

\subsection{Aritméticas y lógicas}

Operación $"y"$ lógico:

\begin{table}[H]
    \centering
    \begin{tabular}{cl}
        \large \textbf{Código intermedio} & \large \textbf{Código objeto} \\ 
        \hline  & \\[-2mm]
        [=and, op1, op2, res] 
        & store\_in\_reg(op1, .R1, Value) \\
        & store\_in\_reg(op2, .R2, Value) \\
        & store\_in\_reg(res, .R3, Dir) \\
        & AND .R1, .R2 \\
        & MOVE .A, [.R3]
        \vspace{2mm} \\
        \hline 
    \end{tabular}
\end{table}

Operación post-auto-decremento:

\begin{table}[H]
    \centering
    \begin{tabular}{cl}
        \large \textbf{Código intermedio} & \large \textbf{Código objeto} \\ 
        \hline  & \\[-2mm]
        [=and, op1, op2, res] 
        & store\_in\_reg(op1, .R1, Value) \\
        & store\_in\_reg(op2, .R2, Value) \\
        & store\_in\_reg(res, .R3, Dir) \\
        & SUB .R1, .R2 \\
        & MOVE .A, [.R3]
        \vspace{2mm} \\
        \hline 
    \end{tabular}
\end{table}

\subsection{Asignaciones}
Asignación de un entero o lógico:

\begin{table}[H]
    \centering
    \begin{tabular}{cl}
        \large \textbf{Código intermedio} & \large \textbf{Código objeto} \\ 
        \hline  & \\[-2mm]
        [=EL, op1, , res] 
        & store\_in\_reg(op1, .R1, Value) \\
        & store\_in\_reg(res, .R3, Dir) \\
        & MOVE .R1, [.R3]
        \vspace{2mm} \\
        \hline 
    \end{tabular}
\end{table}

Asignación de una cadena:

\begin{table}[H]
    \centering
    \begin{tabular}{cl}
        \large \textbf{Código intermedio} & \large \textbf{Código objeto} \\ 
        \hline & \\[-2mm]
        [=Cad, op1, , res] 
        & store\_in\_reg(op1, .R1, Dir) \\
        & store\_in\_reg(res, .R3, Dir) \\
        & MOVE .R1, .R3
        \vspace{2mm} \\
        \hline 
    \end{tabular}
\end{table}

\subsection{Saltos y etiquetas}

Añadir una etiqueta:

\begin{table}[H]
    \centering
    \begin{tabular}{cl}
        \large \textbf{Código intermedio} & \large \textbf{Código objeto} \\ 
        \hline & \\[-2mm]
        [:, op1, ,] 
        & op1: NOP
        \vspace{2mm} \\
        \hline 
    \end{tabular}
\end{table}

Salto a una etiqueta:

\begin{table}[H]
    \centering
    \begin{tabular}{cl}
        \large \textbf{Código intermedio} & \large \textbf{Código objeto} \\ 
        \hline & \\[-2mm]
        [goto, , , etiqueta] 
        & BR /etiqueta
        \vspace{2mm} \\
        \hline 
    \end{tabular}
\end{table}

Salto en caso de cierto:

\begin{table}[H]
    \centering
    \begin{tabular}{cl}
        \large \textbf{Código intermedio} & \large \textbf{Código objeto} \\ 
        \hline  & \\[-2mm]
        [if=goto, op1, op2, etiqueta] 
        & store\_in\_reg(op1, .R1, Value) \\
        & store\_in\_reg(op2, .R2, Value) \\
        & CMP .R1 .R2 \\
        & BZ .A, /etiqueta
        \vspace{2mm} \\
        \hline 
    \end{tabular}
\end{table}

\subsection{Paso de parámetros}

\subsection{Llamadas}
 
\subsection{Sentencias de retorno}

\subsection{Alerts e inputs}

\clearpage
\section{Anexo}
A continuación se listan las cinco pruebas solicitadas.


\subsection{Prueba 1 - test del for}
\vspace{2mm}
Contenido de la prueba:
\vspace{2mm}
\begin{changemargin}{+3cm}{+2cm}
    \lstinputlisting[style=JavaScript]{./resources/global-tests/Prueba 1 - Aaron/test-for.js}
\end{changemargin} 
\vspace{2mm}
Contenido del CI:
\vspace{2mm}
\begin{changemargin}{+1cm}{+0cm}
    \lstinputlisting[style=JavaScript]{./resources/global-tests/Prueba 1 - Aaron/test-for.ci}
\end{changemargin} 
\vspace{2mm}
Contenido del CO:
\vspace{2mm}
\begin{changemargin}{+1cm}{-2cm}
    \lstinputlisting[style=JavaScript]{./resources/global-tests/Prueba 1 - Aaron/test-for.ens}
\end{changemargin} 
\vspace{2mm}
Resultado de la ejecución:
\vspace{2mm}
\begin{changemargin}{+1cm}{-2cm}
    \lstinputlisting[style=JavaScript]{./resources/global-tests/Prueba 1 - Aaron/test-for.out}
\end{changemargin} 

\clearpage

\subsection{Prueba 2 - test de operaciones entrada y salida}
\vspace{2mm}
Contenido de la prueba:
\vspace{2mm}
\begin{changemargin}{+3cm}{+2cm}
    \lstinputlisting[style=JavaScript]{./resources/global-tests/Prueba 2 - Aaron/test-inout.js}
\end{changemargin} 
\vspace{2mm}
Contenido del CI:
\vspace{2mm}
\begin{changemargin}{+1cm}{+0cm}
    \lstinputlisting[style=JavaScript]{./resources/global-tests/Prueba 2 - Aaron/test-inout.ci}
\end{changemargin} 
\vspace{2mm}
Contenido del CO:
\vspace{2mm}
\begin{changemargin}{+1cm}{-2cm}
    \lstinputlisting[style=JavaScript]{./resources/global-tests/Prueba 2 - Aaron/test-inout.ens}
\end{changemargin} 
\vspace{2mm}
Resultado de la ejecución:
\vspace{2mm}
\begin{changemargin}{+1cm}{-2cm}
    \lstinputlisting[style=JavaScript]{./resources/global-tests/Prueba 2 - Aaron/test-inout.out}
\end{changemargin} 

\clearpage

\subsection{Prueba 3 - test genérico}
\vspace{2mm}
Contenido de la prueba:
\vspace{2mm}
\begin{changemargin}{+3cm}{+2cm}
    \lstinputlisting[style=JavaScript]{./resources/global-tests/Prueba 3 - Aaron/test-random.js}
\end{changemargin} 
\vspace{2mm}
Contenido del CI:
\vspace{2mm}
\begin{changemargin}{+1cm}{+0cm}
    \lstinputlisting[style=JavaScript]{./resources/global-tests/Prueba 3 - Aaron/test-random.ci}
\end{changemargin} 
\vspace{2mm}
Contenido del CO:
\vspace{2mm}
\begin{changemargin}{+1cm}{-2cm}
    \lstinputlisting[style=JavaScript]{./resources/global-tests/Prueba 3 - Aaron/test-random.ens}
\end{changemargin}
\vspace{2mm}
Resultado de la ejecución:
\vspace{2mm}
\begin{changemargin}{+1cm}{-2cm}
    \lstinputlisting[style=JavaScript]{./resources/global-tests/Prueba 3 - Aaron/test-random.out}
\end{changemargin} 

\clearpage

\subsection{Prueba 4 - test recursivo}
\vspace{2mm}
Contenido de la prueba:
\vspace{2mm}
\begin{changemargin}{+3cm}{+2cm}
    \lstinputlisting[style=JavaScript]{./resources/global-tests/Prueba 4 - Daniel/test-rec.js}
\end{changemargin} 
\vspace{2mm}
Contenido del CI:
\vspace{2mm}
\begin{changemargin}{+1cm}{+0cm}
    \lstinputlisting[style=JavaScript]{./resources/global-tests/Prueba 4 - Daniel/test-rec.ci}
\end{changemargin} 
\vspace{2mm}
Contenido del CO:
\vspace{2mm}
\begin{changemargin}{+1cm}{-2cm}
    \lstinputlisting[style=JavaScript]{./resources/global-tests/Prueba 4 - Daniel/test-rec.ens}
\end{changemargin}
\vspace{2mm}
Resultado de la ejecución:
\vspace{2mm}
\begin{changemargin}{+1cm}{-2cm}
    \lstinputlisting[style=JavaScript]{./resources/global-tests/Prueba 4 - Daniel/tes-rec.out}
\end{changemargin} 

\clearpage

\subsection{Prueba 5 - test llamadas a funciones}
\vspace{2mm}
Contenido de la prueba:
\vspace{2mm}
\begin{changemargin}{+3cm}{+2cm}
    \lstinputlisting[style=JavaScript]{./resources/global-tests/Prueba 5 - Alejandro/test-nested-functions.js}
\end{changemargin} 
\vspace{2mm}
Contenido del CI:
\vspace{2mm}
\begin{changemargin}{+1cm}{+0cm}
    \lstinputlisting[style=JavaScript]{./resources/global-tests/Prueba 5 - Alejandro/test-nested-functions.ci}
\end{changemargin} 
\vspace{2mm}
Contenido del CO:
\vspace{2mm}
\begin{changemargin}{+1cm}{-2cm}
    \lstinputlisting[style=JavaScript]{./resources/global-tests/Prueba 5 - Alejandro/test-nested-functions.ens}
\end{changemargin} 
\vspace{2mm}
Resultado de la ejecución:
\vspace{2mm}
\begin{changemargin}{+1cm}{-2cm}
    \lstinputlisting[style=JavaScript]{./resources/global-tests/Prueba 5 - Alejandro/test-nested-functions.out}
\end{changemargin} 

\clearpage

\printbibliography[heading=bibnumbered]



\end{document}
